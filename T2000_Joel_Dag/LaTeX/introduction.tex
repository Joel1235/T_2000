% !TEX root =  master.tex
\chapter{Einleitung}

Die Bearbeitung dieser Arbeit findet im betrieblichen Umfeld der Firma Atos Information Technology GmbH am Standort Paderborn in der Abteilung Offensive-Defensive Security (ODS) im Bereich des Penetration Testings statt.

\section{Einordnung des Themas}

Im Zuge der fortschreitenden Digitalisierung und dem zunehmenden Gebrauch von Internet-basierten Anwendungen, haben IT-Systeme einen zentralen Stellenwert in vielen Bereichen des täglichen Lebens und der Geschäftswelt erlangt. Sie ermöglichen die Verwaltung von Geschäftsprozessen, die Kommunikation mit Kunden und die Speicherung von wichtigen Daten. \\
Allerdings erhöht die zunehmende Verbreitung von IT-Systemen auch die Gefahr von Angriffen. Angreifer nutzen häufig Schwachstellen aus, um unbefugte Zugänge zu erhalten und sensiblen Daten für ihre eigenen Zwecke zu missbrauchen.  \\
Häufig werden dabei Schäden verursacht, die Kosten in Millionenhöhe zur Folge haben. Durch die starke Abhängigkeit von IT-gestützen Systemen werden sogar ganze Infrastrukturen aufgrund von Cyberangriffen stillgelegt.

Um solche Systeme zu Schützen und mögliche Sicherheitslücken aufzudecken, werden in der Regel Penetration Tests (Pentests) aus der Sicht eines Angreifers durchgeführt. Die Pentester nutzen dabei sowohl manuell, als auch automatisierte Vorgehensweisen.

Diese Arbeit beschäftigt sich mit dem Thema der automatisierten Erkennung und Ausnutzung von Schwachstellen in Web-Applikationen. Hierbei werden verschiedene Tools und Techniken untersucht und evaluiert, die verwendet werden um Schwachstellen in Web-Applikationen zu erkennen und auszunutzen.

Dadurch soll ein Überblick über verschiedene Methoden und Tools zur Erkennung von Schwachstellen in Web-Applikationen und dessen Vor- und Nachteile geschaffen werden. Auf Basis der Untersuchungen dieser Arbeit soll eine Empfehlung für eine effektive Vorgehensweise bei der automatisierten Erkennung und Ausnutzung von Schwachstellen abgegeben werden. 




\section{Problemstellung}

Die monetäre Kosten, die jährlich durch Cyberattacken entstehen, sind beträchtlich und steigen stetig an. Sie sind mittlerweile vergleichbar mit Ausgaben ganzer Staatshaushälter einzelner Länder. Der Einfluss von Cyberattacken umfasst nicht nur Einzelpersonen, Unternehmen und öffentliche Einrichtungen, sonder auch die gesamte Infrastruktur, auf der zahlreiche Prozesse unserer modernen Gesellschaft beruhen.

In den letzte 10 Jahren hat sich de Zahl der Cyberangriffe in Deutschland rapide erhöht. Laut Bundesamt für Sicherheit in der Informationstechnik (BSI) gab es im Jahr 2018 mehr als 400.000 gemeldete Cyberangriffe, was eine Steigerung von 37\% im Vergleich zum Vorjahr entspricht. \cite{1} \\
Auch die Schäden, die durch Cyberangriffe verursacht werden, haben zugenommen. Laut der aktuellen Ausgabe der Cost of Data Brach Study betrugen die durchschnittlichen Kosten einer Datenpanne in Deutschland 3,62 Millionen US-Dollar. \cite{2} \\
Dabei ist wichtig zu beachten, dass viele Cyberangriffe nicht gemeldet werden und daher die tatsächlichen Zahlen wahrscheinlich höher sind. Außerdem haben sich die Methoden, mit denen Cyberkriminelle Angriffe durchführen, in den letzten Jahren erheblich fortentwickelt, was es für Unternehmen und Organisationen immer schwieriger macht, ihre Systeme zu schützen.

Bei der Erkennung von Schwachstellen spielen Pentesting-Tools eine besonders wichtige Rolle. Unter der zahlreichen Auswahl verschiedener Tools, die bestimmte Funktionen automatisieren, verliert man schnell den Überblick. Daher ist nur schwer ersichtlich, welche Praktiken und zugehörige Tools geeignet sind um automatisiert Schwachstellen in Web-Applikationen zu erkennen und auszunutzen. 




\section{Ziel der Arbeit}

Da alle Tools unterschiedliche Funktionalitäten und Ansätze bieten Schwachstellen zu erkennen, ist es wichtig eine umfassende Untersuchung dieser Tools durchzuführen, um so verschiedene Praktiken bei der Erkennung und Ausnutzung von Schwachstellen zu evaluieren. \\
Das Ziel dieser Arbeit ist es, ein Verständnis dafür zu entwickeln, wie diese Techniken funktionieren und wie sie optimiert werden können.
Dabei konzentriert sich das Vorhaben auf die Identifikation der besten Methoden und Tools, um Schwachstellen zu erkennen und zu beseitigen. 
Durch die Untersuchungen dieser Arbeit sollen praktische Lösungen für die Automatisierung der Schwachstellenanalyse in Web-Applikationen bereitgestellt werden, um so die Sicherheit dieser Anwendungen für die Nutzer zu erhöhen. Zudem sollen neue Ansätze und Methoden vorgestellt werden, die dabei helfen können, Schwachstellen effizienter zu erkennen und zu beseitigen. \\
Insgesamt soll dadurch die zielgerichtetere Erkennung von Schwachstellen in Web-Applikationen erhöht werden, wodurch die Schadensanfälligkeit von Web-Applikationen verringert werden soll. 

\section{Vorgehensweise}

Um eine systematische Vorgehensweise zu garantieren, wird zunächst eine Literaturrecherche durchgeführt, wodurch eine Grundlage für den weiteren Verlauf der Arbeit geschaffen wird. 
Hierbei wird auch der Grundablauf eines Penetration Testers und die Integration von passenden Tools in die Vorgehensweise thematisiert.
Für die Recherche werden sowohl wissenschaftliche Publikationen als auch praxisorientierte Ressourcen herangezogen.

Danach wird eine Auswahl der zu testenden Tools und Techniken. Es werden dazu Kriterien, eine Vorgehensweise und eine passende Testumgebung festgelegt, um die Untersuchung unter möglichst realen Bedingungen vorzunehmen. \\
Anschließend wird die Analyse der unterschiedlichen Tests unter Berücksichtigung der Literaturrecherche durchgeführt. \\
Die hier entstandenen Ergebnisse werden daraufhin durch die aufgestellten Kriterien evaluiert, wodurch eine umfassende Bewertung der verschiedenen Methoden und Tools ermöglicht wird.

Abschließend werden die Ergebnisse der Literaturrecherche und der praktischen Evaluierung zusammengeführt, um eine detaillierte Validierung der verschiedenen Tools und Techniken zu ermöglichen. Hierbei werden auch mögliche Implikationen für die Praxis diskutiert, um Empfehlungen für die Anwendung von Tools und Techniken zur automatisierten Erkennung von Schwachstellen in Web-Applikationen abzugeben.

\section{Aufbau der Arbeit}

Die folgende Arbeit gliedert sich in sieben Bestandteile. Dabei bildet dieser Abschnitt die Einleitung. Hier wird ein Überblick zu dem Thema angefertigt und die
Zielsetzung der Arbeit unter Berücksichtigung der Problemstellung definiert.

Diese Arbeit bewegt sich im praktischen Umfeld des Penetration Testing und der Schwachstellenanalyse von Web-Applikationen.
Daher wird im Teil zwei der Arbeit zunächst eine wissenschaftliche Grundlage zu diesen
Thematiken geschaffen.

Das dritte Kapitel beschäftigt sich mit der Auswahl der zu analysierenden Tools und Techniken. Außerdem wird hier eine Vorgehensweise und eine Testumgebung festgelegt, um optimale Bedingungen und eine systematische Vorgehensweise zu garantieren.

Kapitel vier befasst sich mit der Analyse der ausgewählten Tools, die zur automatisierten Erkennung von Schwachstellen in Web-Applikationen eingesetzt werden. Hierbei werden Funktionalitäten, Vorgehensweisen und Ergebnisse untersucht und verglichen. 

Im darauffolgenden Teil, dem Fünften, wird sich mit der automatisierten Erkennung und Ausnutzung von Schwachstellen in Web-Applikationen unter Verwendung von Künstlicher Intelligenz beschäftigt. Dabei werden jüngste Entwicklungen aus der Forschung berücksichtigt.

Das sechste Kapitel thematisiert die Evaluierung der verschiedenen Tools und Techniken und dient dazu, Stärken und Schwächen der einzelnen Ansätze herauszuarbeiten. Hierbei wird auch ein Vergleich der verschiedenen Tools und Techniken anhand festgelegter Kriterien durchgeführt.

Im letzten Teil der Arbeit werden die Ergebnisse zusammengefasst und evaluiert. Darüber hinaus wird eine Empfehlung für die automatisierte Erkennung und Ausnutzung von Schwachstellen in Web-Applikationen abgegeben.

